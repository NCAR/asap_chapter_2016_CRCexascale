\section{Introduction}\label{sec:intro}

%1) general earth system modeling and why it's important
Climate results from complex interactions between processes in the ocean, atmosphere, land, and ice, and understanding how the Earth's climate system works has long been of scientific interest.  We are currently in the midst of a changing climate, and the need to both explain past and present climate states and to confidently predict future climate states is critical.  In particular, future climate predictions may have societal and policy implications in such areas as water resource planning, agriculture practices, and human activities (e.g., deforestation, $CO_2$ emissions) \cite{washington2005}. Fortunately, advances in computing capabilities over the last decades have enabled the development of climate simulation codes, or Earth-system models, which have significantly advanced climate science and improved understanding of the complexities of the climate system \cite{washington2009}. 

%2)narrow on good of models
Earth system models (ESMs) are particularly important in climate science research as they serve as a virtual laboratory for investigating the effects of external influences, such as increasing $CO_2$ levels in the atmosphere, or more generally answering ``What if ...?'' questions, such as ``What if the Greenland ice sheet melts?'' (e.g., see \cite{heavens2013}, \cite{zeebe2011}, \cite{easterbrook2009}).  In fact, ESMs are increasingly being utilized to better prepare for future climate scenarios, and a prime example are the Coupled Model Comparison Projects (CMIPs) that occur every few years.  The CMIP projects 
are international efforts to coordinate and compare climate model experiments from as many as twenty modelling groups (e.g., CMIP Phase 5 \citep{cmip5}), and the results
% After each CMIP, the Intergovernmental Panel on Climate Change (IPCC) \cite{ipcc-web} releases an assessment report summarizing the findings from the CMIP data, and the IPCC reports (and CMIP data)
%(e.g., CMIP Phase 5 \citep{cmip5} experiments produced the report in \cite{stocker2013ipcc}). The CMIP data and IPCC reports 
are widely utilized by scientists and policymakers alike to inform and prepare for future climate states. Indeed, climate science has a strong computational component that has prominently contributed to our current knowledge of the Earth's climate system.


%3 cimate software in general + eqns and numerics
To be both effective and accurate, climate simulation models must well-represent the physical processes of the major components in the climate system (atmospheres, oceans, sea ice, and land) and capture the complicated interactions between them.  Current climate models typically consist of models for each major climate component that are coupled together such that feedback occurs between components (e.g., \cite{washington2005}).  The model equations for the dynamics of the atmosphere, ocean, and sea ice components are derived from basic conservation laws for energy, mass, and momentum which are tailored for each component.  In addition, an equation of state (i.e., gas law) is needed by the climate model to relate the temperature, pressure and density quantities. 
%(An excellent and accessible resource for the mathematics and physics behind a climate model can be found in \cite{washington2005}.)  
The complexity of the Earth's climate system has generally translated to climate simulations codes that are similarly complex, large in size, and typically the result of many years or even decades of development (e.g., \cite{easterbrook2011},\cite{pipitone2012}. 

Large climate model codes continue to grown even larger and more complex as they evolve to include new science features and take advantage of advances in high performance computing technologies.
% and increases in high performance computing capabilities.
In particular, advances in computing over the last couple decades have allowed for finer resolutions, the inclusion of more physical processes, and improved boundary interactions (e.g., air-sea) \cite{washington2009}. Yet despite steadily increasing high performance computing capabilities, the climate modelling community's requests for computing resources continue to outpace available resources. High-resolution (spatial and temporal) models are increasingly in demand as scientists seek to produce more accurate and realistic simulations, the importance of which should not be underestimated in light of potential societal implications.  High resolution simulations are often required to resolve particular phenomena; for example, ocean eddies, narrow ocean currents, tropical cyclones, fronts, and hurricanes all require finer resolutions than can be reasonably run for multi-century simulations and ensemble studies \cite{washington2009, small2014}.  Therefore in practice, climate scientists must carefully balance their science needs (e.g., resolution, simulation length, process details) with the amount of available computing power and compromises are frequently required \cite{washington2009}.  Given current limitations, if climate model codes are adapted to successfully run on future exascale machines, the potential gains for climate science knowledge may very well be transformative.

%A little about CESM
In this chapter, we discuss the preparation for exascale of a widely used and well-respected climate model: the Community Earth System Model (CESM) \cite{cesm2013}.  CESM is a fully-coupled climate simulation code with a large complex code base (currently over one and a half million lines) that has been developed over the last twenty years.  While its active development is led by the National Center for Atmospheric Research (NCAR), CESM has become increasingly popular with scientists around the globe and is a key contributor to the aforementioned CMIP projects. 
The path to exascale for CESM contains many challenges, and our goal in this chapter is to provide an overview of our current efforts and plans toward preparing CESM for exascale computing, focusing specifically on the CESM software. Clearly, we cannot detail all modifications and decisions made thus far (nor would the reader wish us to do so), and instead we describe our overal optimization methodology and strategy, sharing details via several illustrative examples.

%generic paragraph about this chapter
This Chapter is organized as follows. In Section \ref{sec:algorithm}, ... [{\color{red} Allison:} need to complete]
