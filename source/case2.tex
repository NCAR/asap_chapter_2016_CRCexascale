\section{Case study: physics calculations}\label{sec:phys}

While this case study is of much smaler scale that that in Section \ref{sec:case1}, it is important nonetheless and is representative of on-going efforts ti accelerate the physics modules of CAM.


[{\color{red} Clean this up and tie to big picture}]

binterp is an interpolation routine used in CESM under the CAM module. binterp is a subroutine accepting a little more than ten arguments and has roughly a hundred lines of code including comments. The  lines of code included three doubly nested loop and several branching statements both within and outside the nested loops. Thus, the default execution on Sandybridge processor took about a few hundred nano seconds. Since the routine was called thousands of time, optimizing the kernel was essential.
On profiling the code, it was observed that none of the loops vectorized due to branching statements within the loop. Moreover, more than 70\% of the execution was spent on one single nested for loop. This was because a) the nested loops didn't provision cache blocking and b) there were indexed look-up table reads. Steps adopted to optimize the code- a) the loops were refactored to separate the branches and to accommodate cache blocking, b) multiple look-up table reads of the same array variable were replaced with new copies that had coalesced access of elements. and c) additional flags/directives were added to align arrays and to force vectorization on the loops. 
The optimized binterp, using intel 15.0.3 compiler, was found to provide 3.17x and 1.31x speedup on Sandybridge and Haswell architectures respectively. 

Radiation includes a number of routines, which compute cloud optical properties using CAM method within CESM. The radiation code is divided into two parts based on the solar spectrum as the longwave (LW) and shortwave (SW) radiation code. Each division of the radiation code has three algorithmic parts a) Stochastic array generator, b) Gas optics and c) Solver. The radiation code collectively has more than ten thousand lines of code including. Static analysis of the code showed that a) large number of loop independent computations were repeated within loops, b) several unused copies of arrays were created, and c) no cache blocking strategies were used. 
Profiling of the code revealed that none of the main loops vectorized in the radiation code due to branching statements within the loops. Steps adopted for optimization:
\begin {itemize}
\item {Loop independent operations were pushed outside the loop}
\item {Unused variables were removed}
\item {To facilitate cache blocking, the loops had to be pushed one to two levels lower, which required promoting of variables to higher dimensions and re-coding of routines to accommodate loops}
\item {Loops were split to separate out branches, facilitating vectorization}
\item {Redundant division operations were replaced with reciprocal multiplications}
\end{itemize}

For the compilation, additional flags/directives were added to align arrays, force vectorization on the loops and modify precision models. 
The optimized radiation code, using intel 15.0.3 compiler and MKL 11.2.3, was found to provide 1.62x speedup on Sandybridge architecture. 

The optimizations performed on these modules collectively reduced the cost of CAM by approximately 4.7\%. Estimating NCAR’s Yellowstone core-hour budget for CMIP6’s baseline experiments to be 600 MCPU-hrs, and that CAM itself represents about half of that cost, then these optimizations save roughly 14 MCPU-hrs in CMIP6 costs alone. 

{\color{red} [Need to add material on implicit chemistry solver, Mickelson 1 paragraph]}
