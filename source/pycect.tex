\subsection{CESM Ensemble Verification [1.5 pages, Baker]}\label{sec:pycect}

%problem
The Community Earth System Model (CESM), like most climate simulation
codes, is large and complex.  Further, the CESM code is continually
evolving to accommodate recent science developments, to port to new
HPC platforms requested by the climate community, and to prepare for
future (e.g., exascale) machine architectures, for example. Given its
ongoing state of development, quality assurance via software
verification is particularly critical for CESM to both detect and
reduce errors, thereby ensuring user confidence in the simulation code
and its output. Verifying the correctness of a change or update to the
CESM hardware/software stack (e.g., a new compiler flag, a difference
machine, a code modification) is trivial when the simulation results
after the update are bit-for-bit (BFB) identical with the original
results.  At issue is when the new results are no longer BFB with the
original results.  The chaotic nature of climate model simulations ensures that this
occurrence is commonplace in the CESM development cycle, as tiny code
modifications or a compiler change can easily result in non-BFB solutions
(e.g., see \cite{milroy2016}).

%solution (general) + benefit
When the CESM simulation results have changed due to modification(s)
to the hardware or software stack, determining whether the difference
is significant (i.e., ``climate-changing'') is key.  In particular, we
consider a modification to be admissible if it is statistically
indistinguishable from the original results.  Until recently, making
such a determination was a computationally expensive and subjective
task that required climate expertise (e.g., running and comparing
multi-century simulations).  However, the CESM Ensemble Consistency
Test (ECT) tool introduced \cite{baker2015} had formalized and
simplified an important aspect of CESM quality assurance and is now
regularly used by the CESM software engineers when porting to new
CESM-supported machines and releasing code updates, for
example. Further, because CESM-ECT is objective in nature and
accessible to model developers and users, the tool has proven its
utility in detecting errors and providing rapid feedback to software
developers, thus boosting model confidence.  In particular, the
capability to determine consistency between simulation results that
are not BFB provides the much-needed flexibility, for example, to
pursue more aggressive code optimizations and utilize heterogeneous
execution environments, both of which are critical in the path to
exascale.


%solution (details: module variants, example)
As its name implies, the CESM-ECT tool uses an ensemble-based approach
to determine climate consistency (i.e., statistical
indistinguishability).  In particular, an ensemble of ``accepted''
simulations that represent the same earth system model allows us to
gauge the natural variability of the model’s climate by providing a
qualitative measurement of variability with which to compare future
simulations. The CESM-ECT issues a pass or fail for new non-BFB CESM
output (e.g., from a new machine) by comparing it to the variability
represented by the accepted ensemble and determining whether it is
statistically distinguishable.  The CESM-ECT is a suite of tools
tailored for individual CESM component, and, at present, ECT modules
have been developed and released for the Community Atmosphere Model
(CAM) component (CAM-ECT) \cite{baker2015} and ocean component
(CESM-POP) (POP-ECT) \cite{baker2016}. Note that ECT modules can in
some situations detect errors from other components (e.g., the example
in \cite{baker2015} where CAM-ECT detects an error in the sea ice
component). 

%specific approach details for CAM and POP
While all CESM-ECT modules rely on ensembles, the underlying testing
algorithms may be surprisingly different due characteristics of the
respective models.  For example,CAM-ECT characterizes the ensemble's
distribution by performing principal component (PC) analysis on the
global area-weighted means of 120 CAM variables from an ensemble of
151 one-year climate simulations (that differ only in an initial
round-off level perturbation to the atmosphere temperature), and, from
that, creates a distribution of accepted PC scores.  CAM-ECT then
compares three one-year runs with a modification (e.g., run on a new
platform) with the accepted ensemble distribution.  If more than a
threshold number of PCs are outside of the distribution, then the new
run is deemed to be inconsistant, or statistically distinguishable
from the original (see \cite{baker2015} for further details).  Because
CAM and POP have differing characteristics in term of dynamics,
spatial variability, and timescales, the CAM-ECT approach is not
suitable for POP-ECT.  In contrast, POP-ECT evaluates ensemble means
and deviations for five independent diagnostic variables in a spatial
manner, resulting in a characterization of ensemble distribution at
each grid point (see \cite{baker2016} for further details).
Developing modules for the other components is work in
progress.

%summary (coarse grain) + segue into KGEN section
While CESM-ECT has greatly contributed to quality assurance for CESM,
at present its scope is limited to coarse-grain verification.  In
other words, while CESM-ECT can readily indicate that a problem exists
by issuing a ``fail'', an easy means of identifying the root cause of
the failure is not provided.  For example, for a failing test, CAM-ECT
reports the PCs that differ from the ensemble distribution by more
than a specified threshold amount. Because each PC is a linear
combination of all 120 CAM variables, tracing the failure to a
specific variable is non-trivial (e.g. see Section 6 in
\cite{milroy2016}) and identifying the error in a complex code with
more than one million lines can be daunting.  At least two additional
capabilities are needed to more easily trace a CESM-ECT failure to its
root cause.  First, a utility is needed to identify potential
problematic CESM kernels based on information from the CESM-ECT
failure.  Second, once the kernel(s) have been identified, a utility
is needed to extract kernels from CESM (along with its relevant data)
to render the debugging task reasonable.  While we have not begun work
on the first tool, a tool that fulfills the second capability is
described in the following section.  Finally, we note that our latest
efforts have been focused on the development of an ultrafast version
of CESM-ECT, which only requires nine model steps (equivalent to 4.5
hours in model time) and which has shown promising preliminary
results.
