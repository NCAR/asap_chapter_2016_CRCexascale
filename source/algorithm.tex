
\section{Spectral element dynamical core}\label{sec:algorithm}

%  We focus our optimization efforts on a specific section of the Community Atmosphere Model (CAM) the atmospheric component of CESM that typically consumes greater then 50\% of the total cost of a CESM simulation.  CAM is constructed of a large number of different code modules that are typically  categorized into a dynamics and a physics modules.  The dynamic core solves a set of partial differential equations which describe the fluid flow of the atmosphere, while the physics calculates all other properties.  We describe our work to optimize the spectral element (SE) based dynamical core.  The dynamical core, which can consume 30-88\% of the total cost of the Community Atmosphere Model (CAM), has  a key impact on the overall scalabilty of CAM, and CESM.  While the SE-dynamical core, has demonstrated scaling to approximately 300K hardware cores \cite{came} on a ultra high $1/8^\circ$ degree resolution, simulating climate at such high resolutions is still fundamentally not feasible due to the extreme cost of these simulations a very large core counts.  Even $1/4^\circ$ degree resolution are effectively limited to approximately 10-20K cores due to simulation costs and scalability limitations.  

The High-Order Method Modeling Environment (HOMME) atmospheric dynamical is used by the Community Atmosphere Model (CAM) as the default dynamical core for very high $1/4^\circ$ resolution simulations.   The HOMME dynamical core which can consume 30-88\% of the total cost of the CAM has a key impact on the overall scalability of CAM and CESM.HOMME uses a cubed-sphere topology where each face of the cube is projected onto a sphere to generate a global computational grid.  HOMME uses a spectral element method to discretize in the horizontal and a finite difference approximation \cite{simmons:1981} in the vertical. A continuous Galerkin finite-element method \cite{taylor:1997} is used for the spectral element method. The integrals used in the Galerkin formulation are computed from a Gauss-Lobatto quadrature rule within each element.  HOMME decomposes each time-step into components and the equation, for a compressible fluid with hydrostatic and a shallow water approximations, can be written in terms of a vector U containing the prognostic state variables (velocity, temperature, and surface pressure) as: 
dU/dt = F + D + A + T + R
Where F represents the forcing from physics, D the dissipation, A the dynamics from the primitive equations, T the tracer advection, and R the vertical remapping of the mass and momentum variables. A diagram of the overall computational flow of the HOMME algorithm is provided in Figure \ref{fig:homme-alg}. 

HOMME solves these equations in a time-split fully-explicit form. For time-steps involving the forcing (F) and dissipation (D) terms, a forward Euler time- scheme is used. The dynamics (A) and tracer advection (T) are computed using an N-stage Runge-Kutta time-scheme. The dynamics (A) computes the primitive equations prognostic variables. The tracer advection (T), is based on a finite-volume algorithm and advances the specific humidity, liquid water, ice variables, and additional tracer constituents. For advection, a vertical Lagrangian approach used \cite{lin:2004} where the horizontal advection on Lagrangian vertical levels is followed by remapping (R) the mass and momentum variables back to the reference vertical levels at the end of the time-step.

\begin{figure}[tbp]
 \begin{center}
\includegraphics[width=12.0cm]{figures/HOMME-v01.pdf}
\end{center}
\caption{The computational phases within HOMME.}
\label{fig:homme-alg}
\end{figure}
