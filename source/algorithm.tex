\section{Algorithmic detail [{\color{red} 2 pages, Dennis}]}\label{sec:algorithm}

  We focus our optimization efforts of two specific sections of the CESM code base. The 
 Community Atmosphere Model (CAM) which typically consumes greater then 50\% of the total cost of a typical 
 CESM simulation.  The CAM model is constructed of a large number of different code modules that are typically 
 categorized into a dynamics and a physics modules.  The dynamic core solves a set of partial differential equations
 which describe the fluid flow of the atmosphere, while the physics calculates all other properties.  For example the physics 
 modules would calculate the impact of solar radiation on aerosol particles and their chemical reactions with other 
 atmospheric chemical species.  In this section, we describe work performed to optimize the  spectral element (SE) based
  dynamical core and the implict chemistry solver.  The dynamical core, which can consume 30-88\% of the total cost of the Community Atmosphere Model (CAM), has  a key impact on the overall scalabilty of CAM, and CESM.  For the dynamical core we focus on 
both increasing it's single threaded performance as well as increasing the amount of threads it can utilize.  
The implicit chemistry solver  uses a ??? numerical method to solve a system of chemical equations at each point 
in the atmosphere model.  Depending on the exact scientific configuration of CESM, the cost of the implicit chemistry solver can 
vary from insignificant, to approximately 12\% of the total time.  While less important overall then the dynamical core, we include the 
implicit chemistry solver to illustrate the type of improvements that are possible for physic based modules.  

\input homme-alg.tex
